\documentclass[final]{beamer}
\mode<presentation>

% STEP 1: Change the next line according to your language
\usepackage[english]{babel}

% STEP 2: Make sure this character encoding matches the one you save this file as
% (this template is utf8 by default but your editor may change it, causing problems)
\usepackage[utf8]{inputenc}

% You probably don't need to touch the following four lines
\usepackage[T1]{fontenc}
\usepackage{lmodern}
\usepackage{amsmath,amsthm, amssymb, latexsym}
\usepackage{exscale} % required to scale math fonts properly

% PAUSE: my own packages
% figure numbers, captions and subcaptions
\usepackage[font=scriptsize,format=plain,labelfont=bf,up,textfont=it,up,compatibility=false]{caption}
\usepackage[font=scriptsize,compatibility=false]{subcaption}
\usepackage[caption=false]{subfig}

\setbeamertemplate{caption}[numbered]

% bibliografia
%\usepackage[fixlanguage]{babelbib}
\usepackage[round,sort,nonamebreak]{natbib} % citação bibliográfica textual(plainnat-ime.bst)
%\bibpunct{(}{)}{;}{a}{\hspace{-0.7ex},}{,} % estilo de citação. Veja alguns
% exemplos em http://merkel.zoneo.net/Latex/natbib.php
\bibliographystyle{styles/plainnat-ime} 	% citação bibliográfica textual


% Glossaries
\usepackage[sanitize=none,acronym,toc]{glossaries}
%\usepackage[acronym,toc]{glossaries}
% Define a new glossary type
\newglossary[slg,toc]{symbols}{sym}{sbl}{Lista de Símbolos}
\newglossary[elg]{equations}{eqn}{eql}{Equations}
\makeglossaries


% ---------------------------------------------------------------------------- %
% Dicionario de Termos:
\input{d01-terms}       
\input{d02-symbols}       
\input{d03-acronyms}       
\input{d04-equations}       
% ---------------------------------------------------------------------------- %


\graphicspath{{./images/}}             % caminho das figuras (recomendável)
\include{template}  % THIS is the line that includes the template!


% ---------------------------------------------------------------------------- %
\DeclareMathOperator*{\argmin}{arg\,min}
\DeclareMathOperator*{\argmax}{arg\,max}
\DeclareMathOperator*{\erf}{erf}
% ---------------------------------------------------------------------------- %


% Orientation is set here
\usepackage[orientation=portrait,size=a0,scale=1.4]{beamerposter}

% STEP 3:
% Change colours by setting \usetheme[<id>, twocolumn]{HYposter}.
\usetheme[iag, threecolumn]{HYposter}
% The different ids are:
%  maa: Faculty of Agriculture and Forestry 
%  hum: Faculty of Arts 
%  kay: Faculty of Behavioural Sciences 
%  bio: Faculty of Biological and Environmental Sciences 
%  oik: Faculty of Law 
%  med: Faculty of Medicine 
%  far: Faculty of Pharmacy 
%  mat: Faculty of Science 
%  val: Faculty of Social Sciences 
%  teo: Faculty of Theology 
%  ell: Faculty of Veterinary Medicine 
%  soc: Swedish School of Social Science 
%  kir: University of Helsinki Library
%  avo: Open University
%  ale: Aleksanteri Institute
%  neu: Neuroscience Institute
%  biot: Bioscience Institute
%  atk: Computer centre
%  rur: Ruralia Institute
%  koe: Laboratory animal centre
%  kol: Collegium for Advanced Studies
%  til: Center for Properties and Facilities
%  pal: Palmenia
%  kie: Language centre
% Without options a black theme without faculty name will be used.


% STEP 4: Set up the title and author info
\titlestart{Smoothing techniques applied to} % first line of title
\titleend{Brazilian seismic sources characterization} % second line of title
\titlesize{\LARGE} % Use this to change title size if necessary. See README
% for details.

\author{Marlon Pirchiner$^{1,2}$}
\institute{$^1$Seismological Centre, IAG-USP, \url{marlon@iag.usp.br},
\\$^2$Applied Math School, EMAp-FGV-RJ}

% Stuff such as logos of contributing institutes can be put in the lower left corner using this
\leftcorner{
\includegraphics[width=5cm]{images/qr_code}
}


\begin{document}
\begin{poster}


%===========================================
\newcolumn
%===========================================


% STEP 5: Add the contents of your poster between \begin{poster} and \end{poster}
%-------------------------------------------
\section{Seismicity}
%-------------------------------------------

The brazilian seismicity is shown on figure \ref{fig:br_seis}.
\begin{figure}[H]
	\scriptsize
	\centering
	\includegraphics[width=1.0\textwidth]{seismicity_br} 
	\caption{Brazilian seismicity. \gls{bsb2013} catalogue.}
	\label{fig:br_seis} 
\end{figure}


\begin{figure}[H]
	\centering
	\begin{subfigure}[T]{0.52\textwidth}
	  	\centering
		\includegraphics[width=1.0\textwidth]{hmtk_bsb2013_rate}
		\subcaption{Annual Seismic Rate}
		\label{fig:sa_eq_record}
	\end{subfigure}
	\begin{subfigure}[T]{0.45\textwidth}
	  	\centering
		\includegraphics[width=1.0\textwidth]{occurrence}
		\subcaption{Recurrence}
		\label{fig:br_eq_record}
    \end{subfigure}
	\caption{Time and Magnitude distribution}
	\label{fig:eq_record}
\end{figure}



%-------------------------------------------
\section{Declustering and Completeness}
%-------------------------------------------


\begin{figure}[H]
	\centering
	\begin{subfigure}[T]{0.49\textwidth}
	  	\centering
		\includegraphics[width=0.98\textwidth]{decluster_br}
		\caption{Cumulative EQ records}
		\label{fig:br_eq_record}
	\end{subfigure}
	\begin{subfigure}[T]{0.49\textwidth}
	  	\centering
		\includegraphics[width=0.98\textwidth]{stepp_br}
		\caption{Stepp diagram}
		\label{fig:br_stepp}
    \end{subfigure}%
	\caption{Declustering and completeness magnitude evaluation}
	\label{fig:eq_record}
\end{figure}



%-------------------------------------------
\section{Previous work}
%-------------------------------------------


\begin{figure}[H]
  \centering
	\begin{subfigure}[t]{0.49\textwidth}
	  \centering
	  \includegraphics[width=1\textwidth]{pga_gshap} 
	  \caption{GSHAP \gls{PGA} (10\%/50anos) [$g$]}
	  \label{fig:gshap} 
	\end{subfigure}
	\begin{subfigure}[t]{0.49\textwidth}
	  \centering
	  \includegraphics[width=1\textwidth]{pga_dourado_oq} 
	  \caption{Mapa de ameaça sísmica, PGA(poe 0.1, 50y)[Dourado, 20014]
	  OpenQuake-Engine }
	  \label{fig:pga_dourado_oq} 
	\end{subfigure}
  \caption{Previous results}
  \label{fig:prev_results} 
\end{figure}


%===========================================
\newcolumn
%===========================================

%-------------------------------------------
\section{Frankel, 1995}
%-------------------------------------------

\begin{equation}
	\ensuremath{
		\tilde{n}_j = \frac{ \sum_{i} n_i \,e^{ - \left(\frac{\gls{sym:dij}}{\gls{sym:dF}}\right)^2}}
						   { \sum_{i}     e^{ - \left(\frac{\gls{sym:dij}}{\gls{sym:dF}}\right)^2}},
	}
	\label{eq:ni}
\end{equation}
\footnotesize
onde 
\begin{itemize}
	\item $d_F$ é a largura de banda \alert{fixa}, nomeada distância de
	correlação
	\item $\tilde{n}_j$ é a taxa cumulativa de sismicidade (número de sismos com magnitude
	$m$ maior que a mínima magnitude \gls{sym:Md} do catálogo) suavizada na célula $j$
	\item $n_i$ é o número de sismos em cada outra célula $i$ e
	\item \gls{sym:dij} é \glsdesc{sym:dij}.
\end{itemize}
\item 

%-------------------------------------------
\section{Woo, 1996}
%-------------------------------------------
\small
		\begin{equation}
			\ensuremath{
				\gls{sym:Rrm} = \sum_{i=1}^{N} \frac{ K(\gls{sym:r} - \gls{sym:ri}, m)}
													{T({\gls{sym:ri})}},
			}
			\label{eq:Rrm}
		\end{equation}
\footnotesize
	onde $N$ é o número de tremores $i$ no catálogo 
	e $T(\gls{sym:ri})$ é o período em que todo sismo de magnitude acima de $m$ é completamente observado 
	em \gls{sym:ri}
\small
		\begin{equation}
			\ensuremath{
				K_{KJ}(\gls{sym:r}, m \arrowvert \gls{sym:aW}) =  \frac{  \gls{sym:aW}  -1}{\pi\gls{sym:hm}^2}
									\left( 1 + \frac{\gls{sym:r}^2}{\gls{sym:hm}^2} \right)^{-\gls{sym:aW}},
			}
			\label{eq:k_kj}
		\end{equation}
\footnotesize
	onde \gls{sym:aW} é \glsdesc{sym:aW}


\small
	\begin{equation}
		\ensuremath{
			h(m\arrowvert \gls{sym:a0}, \gls{sym:a1}) = \gls{sym:a0}e^{\gls{sym:a1}m},
		}
		\label{eq:hm}
	\end{equation}
\footnotesize
$a_0$ e $a_1$ são determinados pela regressão entre a 
distância média $h$ de cada tremor ao vizinho mais próximo em cada faixa de magnitude $m \pm \mathrm{d}m$

\begin{figure}[H]
  \centering
  \includegraphics[width=.5\textwidth]{woo_bandwidth} 
  \caption{Ajuste da largura de banda para o método de Woo1996}
  \label{fig:woo_b} 
\end{figure}


%-------------------------------------------
\section{Helmstetter, 2012}
%-------------------------------------------
\small
		\begin{equation}
		\ensuremath{\gls{sym:R} = \gls{sym:Rmin} + \sum_{t_i < t}{ 
			\frac{2\,w(\boldsymbol{r}_i,t_i)}{h_i\,{d_i}^2}
					\gls{sym:Kt}\gls{sym:Kr} }},
			\label{eq:helms02}
		\end{equation}
\footnotesize
	onde \gls{sym:Rmin} é a \glsdesc{sym:Rmin} e os

\small
		\begin{equation}
			\ensuremath{ w(\boldsymbol{r},t) = 10^{ \gls{sym:b}(\boldsymbol{r},t) \left[ \gls{sym:Mc_rt} - \gls{sym:Md}
			\right] } },
			\label{eq:helms_wi}
		\end{equation}
\footnotesize
	onde \gls{sym:wi} é o \glsdesc{sym:wi} na localização $\boldsymbol{r}$ e no instante $t$, 
		  \gls{sym:b}$(\boldsymbol{r},t)$ é o \glsdesc{sym:b}, 
		  \gls{sym:Mc_rt} é a \glsdesc{sym:Mc_rt}, 
		  \gls{sym:Md} é a \glsdesc{sym:Md}.


\small
		\begin{equation}
			\ensuremath{
		%		h_i, d_i = \underset{d_i \ge \gls{sym:dk}, h_i \ge \gls{sym:hk}}{\argmin} 
				h_i, d_i = \argmin_{\substack{h_i \ge \gls{sym:hk} \\
								              d_i \ge \gls{sym:dk}}
						           } 
				\left[ s \left(h_i,d_i 
					 		  \arrowvert
							  \gls{sym:k_cnn},\gls{sym:a_cnn}
					     \right) 
					   := h_i + \gls{sym:a_cnn}d_i 
			    \right],
			}
			\label{eq:helms_cnn}
		\end{equation}
\footnotesize
	onde \gls{sym:k_cnn} é o \glsdesc{sym:k_cnn},
		 \gls{sym:a_cnn} é o \glsdesc{sym:a_cnn},
		 \gls{sym:dk} é o \glsdesc{sym:dk} e 
		 \gls{sym:hk} é o \glsdesc{sym:hk}.



\footnotesize
		\begin{equation}
			\ensuremath{
				\bar{R}(\boldsymbol{r}_0) = \text{Mediana}\left[R(\boldsymbol{r}_0, t)\right].
			}
			\label{eq:helms_mediana}
		\end{equation}


\begin{figure}[H]
\begin{subfigure}[T]{0.49\textwidth}
  \centering
  \includegraphics[width=.98\textwidth]{helmstetter_hidi} 
  \subcaption{Exemplo da largura de banda para um determinado evento para o
  método de Helmstetter, com $k_{cnn} = 5$ e $a_{cnn} = 100$}
  \label{fig:h_hidi} 
\end{subfigure}
\begin{subfigure}[T]{0.49\textwidth}
  \centering
  \includegraphics[width=.98\textwidth]{helmstetter_stationary_a} 
  \subcaption{Taxa de sismicidade estacionaria calculada a partir da mediana da
  taxa de sismicidade modelada pelo método de Helmstetter2012 para uma determinada célula $r_0$}
  \label{fig:h_stationary} 
\end{subfigure}
\end{figure}



%===========================================
\newcolumn
%===========================================


\small
		\begin{equation}
			\ensuremath{
				\gls{sym:pNn} = \frac{{N_p}^n e^{-N_p}}{n!}.
			}
			\label{eq:loglik}
		\end{equation}

		\begin{equation}
			\ensuremath{
				\gls{sym:L} = \sum_{i_x=1}^{N_x}\sum_{i_y=1}^{N_y}\log p\left[  \gls{sym:Np}, \gls{sym:nxy}  \right]
			}
			\label{eq:loglik}
		\end{equation}
\footnotesize
	onde \gls{sym:Np} é \glsdesc{sym:Np},
		\gls{sym:nxy} é \glsdesc{sym:nxy}.

\small
		\begin{equation}
			\ensuremath{
			\begin{align}
				G & = e^{\sum_{i = 1}^{N_t}
							\frac{\log \left[  \gls{sym:Npi} / \gls{sym:Nu}  \right]}
								 {\gls{sym:Nt}}
					  } \\
				  & = {\langle  \gls{sym:Npi} / \gls{sym:Nu}  \rangle}_{geom}
			\end{align}}
			\label{eq:G}
		\end{equation}	

\vspace{0.3cm}
\begin{columns}[c,totalwidth=\textwidth]
\column{.5\linewidth}

\begin{figure}[H]
  \centering
  \includegraphics[width=.98\textwidth]{helmstetter_catalogues} 
  \caption{Catálogos de aprendizado e de teste para o método de \citet{helmstetter_2012}}
  \label{fig:h_catalogue} 
\end{figure}

\column{.5\linewidth}

\begin{table}[H]
	\centering
	\begin{tabular}{c|c}
		Parâmetro & Valor \\ \hline
		$R_{min}$ & $0.1\times10^{-13}$ \\
		$a_{cnn}$ & 325 \\
		$k_{cnn}$ & 1 \\ \hline
		Ganho	  & 2.43
	\end{tabular}
	\caption{Parâmetros otimizados e Ganho para o método de Helmstetter a partir do catálogo \gls{bsb2013}}
	\label{tab:hemlstetter}
\end{table}


\end{columns}


\vspace{0.3cm}
%-------------------------------------------
\section{Rates and Hazard}
%-------------------------------------------

\begin{columns}[t,totalwidth=\textwidth]
\column{.5\linewidth}



\begin{figure}[H]
  \centering
  \includegraphics[width=.98\textwidth]{a_frankel_br} 
  \caption{Mapa do valor-a, usando o catálogo \gls{bsb2013} calculado pelo método de Frankel, 1995 }
  \label{fig:a_fran_br} 
\end{figure}

\begin{figure}[H]
  \centering
  \includegraphics[width=.98\textwidth]{a_woo} 
  \caption{Mapa do valor-a, usando o catálogo \gls{bsb2013} calculado pelo método de Woo, 1996 }
  \label{fig:a_woo} 
\end{figure}

\begin{figure}[H]
  \centering
  \includegraphics[width=.98\textwidth]{a_helmstetter} 
  \caption{Mapa do valor-a, usando o catálogo \gls{bsb2013} calculado pelo método de Helmstetter, 2012 }
  \label{fig:helm_r} 
\end{figure}



\column{.5\linewidth}



\begin{figure}[H]
  \centering
  \includegraphics[width=.98\textwidth]{pga_frankel} 
  \caption{Mapa de ameaça sísmica, PGA (poe 10\%, 50y) [Frankel, 1995] }
  \label{fig:pga_fran} 
\end{figure}

\begin{figure}[H]
	\centering
	\includegraphics[width=.98\textwidth]{pga_woo_cum} 
	\caption{Mapa de ameaça sísmica, PGA (poe 10\%, 50y).}
	\label{fig:pga_woo_cum} 
\end{figure}

\begin{figure}[H]
  \centering
  \includegraphics[width=.98\textwidth]{pga_helmstetter} 
  \caption{Mapa de ameaça sísmica, PGA (poe 10\%, 50y), 
  		   calculado com o OpenQuake a partir das fontes sísmicas
  		   determinas pelo método de Helmstetter,2012 }
  \label{fig:helm_h} 
\end{figure}



\end{columns}


%-------------------------------------------
\section{References}
%-------------------------------------------
	\scriptsize
%	\bibliography{bib/bibliografia}

\bibitem[Dourado (2014)]{dourado_2014}{\textbf{Dourado (2014)}}
J.C Dourado.
Mapa de amea{\c c}a s{\'\i}smica do brasil.
Em \emph{Congresso Brasileiro de Geologia}.

\bibitem[Frankel (1995)]{frankel_1995}{\textbf{Frankel (1995)}}
Arthur Frankel.
Mapping seismic hazard in the central and eastern united states.
\emph{Seismological Research Letters}, 66\penalty0 (4):\penalty0
  8--21.

\bibitem[Giardini {\rm{\em et~al.}} (1999)Giardini, Gr{\"u}nthal, Shedlock, e
  Zhang]{giardini_1999}{\textbf{Giardini \emph{et~al}} (1999)}
D.~Giardini, G.~Gr{\"u}nthal, K.~M. Shedlock e P.~Zhang.
The {GSHAP} global seismic hazard map.
\emph{Annali di Geofisica}, 42\penalty0 (6):\penalty0 1225--1230.

\bibitem[Helmstetter e Werner (2012)]{helmstetter_2012}{\textbf{Helmstetter e
  Werner (2012)}}
Agn{\`e}s Helmstetter e Maximilian~J. Werner.
Adaptive spatiotemporal smoothing of seismicity for long-term
  earthquake forecasts in california.
\emph{Bulletin of the Seismological Society of America}, 102\penalty0
  (6):\penalty0 2518--2529.

\bibitem[Woo (1996)]{woo_1996}{\textbf{Woo (1996)}}
G.~Woo.
Kernel estimation methods for seismic hazard area source modeling.
\emph{Bulletin of the Seismological Society of America}, 86\penalty0
  (2):\penalty0 353--362.

\end{poster}
\end{document}