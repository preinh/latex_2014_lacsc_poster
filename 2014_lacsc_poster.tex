\documentclass[final]{beamer}
\mode<presentation>

% STEP 1: Change the next line according to your language
\usepackage[english]{babel}

% STEP 2: Make sure this character encoding matches the one you save this file as
% (this template is utf8 by default but your editor may change it, causing problems)
\usepackage[utf8]{inputenc}

% You probably don't need to touch the following four lines
\usepackage[T1]{fontenc}
\usepackage{lmodern}
\usepackage{amsmath,amsthm, amssymb, latexsym}
\usepackage{exscale} % required to scale math fonts properly

% PAUSE: my own packages
% figure numbers, captions and subcaptions
\usepackage[font=footnotesize,format=plain,labelfont=bf,up,textfont=it,up,compatibility=false]{caption}
\usepackage[font=scriptsize,compatibility=false]{subcaption}
\usepackage[caption=false]{subfig}

\setbeamertemplate{caption}[numbered]

% bibliografia
\usepackage[fixlanguage]{babelbib}
\usepackage[round,sort,nonamebreak]{natbib} % citação bibliográfica textual(plainnat-ime.bst)
\bibpunct{(}{)}{;}{a}{\hspace{-0.7ex},}{,} % estilo de citação. Veja alguns exemplos em http://merkel.zoneo.net/Latex/natbib.php


% Glossaries
\usepackage[sanitize=none,acronym,toc]{glossaries}
%\usepackage[acronym,toc]{glossaries}
% Define a new glossary type
\newglossary[slg,toc]{symbols}{sym}{sbl}{Lista de Símbolos}
\newglossary[elg]{equations}{eqn}{eql}{Equations}
\makeglossaries


% ---------------------------------------------------------------------------- %
% Dicionario de Termos:
\input{d01-terms}       
\input{d02-symbols}       
\input{d03-acronyms}       
\input{d04-equations}       
% ---------------------------------------------------------------------------- %


\graphicspath{{./images/}}             % caminho das figuras (recomendável)
\include{template}  % THIS is the line that includes the template!


% ---------------------------------------------------------------------------- %
\DeclareMathOperator*{\argmin}{arg\,min}
\DeclareMathOperator*{\argmax}{arg\,max}
\DeclareMathOperator*{\erf}{erf}
% ---------------------------------------------------------------------------- %



% Orientation is set here
\usepackage[orientation=portrait,size=a0,scale=1.4]{beamerposter}

% STEP 3:
% Change colours by setting \usetheme[<id>, twocolumn]{HYposter}.
\usetheme[iag, threecolumn]{HYposter}
% The different ids are:
%  maa: Faculty of Agriculture and Forestry 
%  hum: Faculty of Arts 
%  kay: Faculty of Behavioural Sciences 
%  bio: Faculty of Biological and Environmental Sciences 
%  oik: Faculty of Law 
%  med: Faculty of Medicine 
%  far: Faculty of Pharmacy 
%  mat: Faculty of Science 
%  val: Faculty of Social Sciences 
%  teo: Faculty of Theology 
%  ell: Faculty of Veterinary Medicine 
%  soc: Swedish School of Social Science 
%  kir: University of Helsinki Library
%  avo: Open University
%  ale: Aleksanteri Institute
%  neu: Neuroscience Institute
%  biot: Bioscience Institute
%  atk: Computer centre
%  rur: Ruralia Institute
%  koe: Laboratory animal centre
%  kol: Collegium for Advanced Studies
%  til: Center for Properties and Facilities
%  pal: Palmenia
%  kie: Language centre
% Without options a black theme without faculty name will be used.


% STEP 4: Set up the title and author info
\titlestart{Smoothing techniques applied to} % first line of title
\titleend{Brazilian seismic sources characterization} % second line of title
\titlesize{\LARGE} % Use this to change title size if necessary. See README
% for details.

\author{Marlon Pirchiner$^{1,2}$}
\institute{$^1$Seismological Centre, IAG-USP, \url{marlon@iag.usp.br},
\\$^2$Applied Math School, EMAp-FGV-RJ}

% Stuff such as logos of contributing institutes can be put in the lower left corner using this
\leftcorner{
\includegraphics[width=5cm]{images/qr_code}
}


\begin{document}
\begin{poster}


%===========================================
\newcolumn
%===========================================


% STEP 5: Add the contents of your poster between \begin{poster} and \end{poster}
%-------------------------------------------
\section{Seismicity}
%-------------------------------------------

The brazilian seismicity is shown on figure \ref{fig:br_seis}.
\begin{figure}[H]
	\scriptsize
	\centering
	\includegraphics[width=1.0\textwidth]{seismicity_br} 
	\caption{Brazilian seismicity. \gls{bsb2013} catalogue.}
	\label{fig:br_seis} 
\end{figure}


\begin{figure}[H]
	\centering
	\begin{subfigure}[T]{0.52\textwidth}
	  	\centering
		\includegraphics[width=1.0\textwidth]{hmtk_bsb2013_rate}
		\subcaption{Annual Seismic Rate}
		\label{fig:sa_eq_record}
	\end{subfigure}
	\begin{subfigure}[T]{0.45\textwidth}
	  	\centering
		\includegraphics[width=1.0\textwidth]{occurrence}
		\subcaption{Recurrence}
		\label{fig:br_eq_record}
    \end{subfigure}%
	\caption{Time and Magnitude distribution}
	\label{fig:eq_record}
\end{figure}



%-------------------------------------------
\section{Declustering and Completeness}
%-------------------------------------------


\begin{figure}[H]
	\centering
	\begin{subfigure}[T]{0.49\textwidth}
	  	\centering
		\includegraphics[width=0.98\textwidth]{decluster_br}
		\caption{Cumulative EQ records}
		\label{fig:br_eq_record}
	\end{subfigure}
	\begin{subfigure}[T]{0.49\textwidth}
	  	\centering
		\includegraphics[width=0.98\textwidth]{stepp_br}
		\caption{Stepp diagram}
		\label{fig:br_stepp}
    \end{subfigure}%
	\caption{Declustering and completeness magnitude evaluation}
	\label{fig:eq_record}
\end{figure}



%-------------------------------------------
\section{Previous work}
%-------------------------------------------


\begin{figure}[H]
  \centering
	\begin{subfigure}[t]{0.49\textwidth}
	  \centering
	  \includegraphics[width=1\textwidth]{pga_gshap} 
	  \caption{GSHAP \gls{PGA} (10\%/50anos) [$g$]}
	  \label{fig:gshap} 
	\end{subfigure}
	\begin{subfigure}[t]{0.49\textwidth}
	  \centering
	  \includegraphics[width=1\textwidth]{pga_dourado_oq} 
	  \caption{Mapa de ameaça sísmica, PGA(poe 0.1, 50y)[Dourado, 20014]
	  OpenQuake-Engine }
	  \label{fig:pga_dourado_oq} 
	\end{subfigure}
  \caption{Previous results}
  \label{fig:prev_results} 
\end{figure}


%===========================================
\newcolumn
%===========================================

\section{Another section}
This is the first section of the second column. The widths of the columns are fixed by either the \texttt{twocolumn} or \texttt{threecolumn} options. Therefore this two-column template should only have two \verb+\newcolumn+ commands. Any more will just mess up the layout. 



%===========================================
\newcolumn
%===========================================



\section{Yet another section}
This is the first section of the third column.



\end{poster}
\end{document}