\documentclass[final]{beamer}
\mode<presentation>

% STEP 1: Change the next line according to your language
\usepackage[english]{babel}

% STEP 2: Make sure this character encoding matches the one you save this file as
% (this template is utf8 by default but your editor may change it, causing problems)
\usepackage[utf8]{inputenc}

% You probably don't need to touch the following four lines
\usepackage[T1]{fontenc}
\usepackage{lmodern}
\usepackage{amsmath,amsthm, amssymb, latexsym}
\usepackage{exscale} % required to scale math fonts properly

% PAUSE: my own packages
% figure numbers, captions and subcaptions
\usepackage[font=footnotesize,format=plain,labelfont=bf,up,textfont=it,up]{caption}
%\usepackage[font=scriptsize]{subcaption}
\usepackage[caption=false]{subfig}

\setbeamertemplate{caption}[numbered]


% Glossaries
\usepackage[sanitize=none,acronym,toc]{glossaries}
%\usepackage[acronym,toc]{glossaries}
% Define a new glossary type
\newglossary[slg,toc]{symbols}{sym}{sbl}{Lista de Símbolos}
\newglossary[elg]{equations}{eqn}{eql}{Equations}
\makeglossaries


% ---------------------------------------------------------------------------- %
% Dicionario de Termos:
\input{d01-terms}       
\input{d02-symbols}       
\input{d03-acronyms}       
\input{d04-equations}       
% ---------------------------------------------------------------------------- %


\graphicspath{{./images/}}             % caminho das figuras (recomendável)
\include{template}  % THIS is the line that includes the template!


% ---------------------------------------------------------------------------- %
\DeclareMathOperator*{\argmin}{arg\,min}
\DeclareMathOperator*{\argmax}{arg\,max}
\DeclareMathOperator*{\erf}{erf}
% ---------------------------------------------------------------------------- %


% Orientation is set here
\usepackage[orientation=portrait,size=a0,scale=1.4]{beamerposter}

% STEP 3:
% Change colours by setting \usetheme[<id>, twocolumn]{HYposter}.
\usetheme[iag, threecolumn]{HYposter}
% The different ids are:
%  maa: Faculty of Agriculture and Forestry 
%  hum: Faculty of Arts 
%  kay: Faculty of Behavioural Sciences 
%  bio: Faculty of Biological and Environmental Sciences 
%  oik: Faculty of Law 
%  med: Faculty of Medicine 
%  far: Faculty of Pharmacy 
%  mat: Faculty of Science 
%  val: Faculty of Social Sciences 
%  teo: Faculty of Theology 
%  ell: Faculty of Veterinary Medicine 
%  soc: Swedish School of Social Science 
%  kir: University of Helsinki Library
%  avo: Open University
%  ale: Aleksanteri Institute
%  neu: Neuroscience Institute
%  biot: Bioscience Institute
%  atk: Computer centre
%  rur: Ruralia Institute
%  koe: Laboratory animal centre
%  kol: Collegium for Advanced Studies
%  til: Center for Properties and Facilities
%  pal: Palmenia
%  kie: Language centre
% Without options a black theme without faculty name will be used.


% STEP 4: Set up the title and author info
\titlestart{Smoothing techniques applied to} % first line of title
\titleend{Brazilian seismic sources characterization} % second line of title
\titlesize{\LARGE} % Use this to change title size if necessary. See README
% for details.

\author{Marlon Pirchiner$^{1,2}$}
\institute{$^1$Seismological Centre, IAG-USP, \url{marlon@iag.usp.br},
\\$^2$Applied Math School, EMAp-FGV-RJ}

% Stuff such as logos of contributing institutes can be put in the lower left corner using this
\leftcorner{
\includegraphics[width=5cm]{images/qr_code}
}

%\usepackage[hyphens]{url}
%\usepackage{breakurl}
%\usepackage[breaklinks]{hyperref}
%\def\UrlBreaks{\do\/\do-}

% bibliografia
\usepackage[fixlanguage]{babelbib}
\usepackage[round,comma,numbers,sort&compress,nonamebreak]{natbib} % citação bibliográfica textual(plainnat-ime.bst)
\bibpunct{(}{)}{;}{a}{\hspace{-0.7ex},}{,} % estilo de citação. Veja alguns
% exemplos em http://merkel.zoneo.net/Latex/natbib.php
%\bibliographystyle{plain} 	% citação bibliográfica textual
\bibliographystyle{styles/plainnat-ime} 	% citação bibliográfica textual


\begin{document}
\begin{poster}


%===========================================
\newcolumn
%===========================================
asdfasd
\cite{woo_1996}


% STEP 5: Add the contents of your poster between \begin{poster} and \end{poster}
%-------------------------------------------
\section{Seismicity}
%-------------------------------------------

The brazilian seismicity is shown on figure 

\citet{frankel_1995}.

\citet{helmstetter_2012}.


%===========================================
\newcolumn
%===========================================


%-------------------------------------------
\section{Frankel, 1995}
%-------------------------------------------

\begin{block}{}
bla 

\bibliography{bib/bibliografia}

bla
\end{block}


asd ASDF

%===========================================
\newcolumn
%===========================================

asdf
%-------------------------------------------
\section{References}
%-------------------------------------------
asdf	
\scriptsize


\bibitem[Frankel (1995)]{frankel_1995}{Frankel (1995)}
Arthur Frankel.
Mapping seismic hazard in the central and eastern united states.
\emph{Seismological Research Letters}, 66\penalty0 (4):\penalty0
  8--21.

\bibitem[Helmstetter e Werner (2012)]{helmstetter_2012}{Helmstetter e
  Werner (2012)}
Agn{\`e}s Helmstetter e Maximilian~J. Werner.
Adaptive spatiotemporal smoothing of seismicity for long-term
  earthquake forecasts in california.
\emph{Bulletin of the Seismological Society of America}, 102\penalty0
  (6):\penalty0 2518--2529.

\bibitem[Woo (1996)]{woo_1996}{Woo (1996)}
G.~Woo.
Kernel estimation methods for seismic hazard area source modeling.
\emph{Bulletin of the Seismological Society of America}, 86\penalty0
  (2):\penalty0 353--362.



	
\end{poster}
\end{document}